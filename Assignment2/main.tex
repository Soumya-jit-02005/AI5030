\documentclass[12pt]{article}
\usepackage[utf8]{inputenc}
\usepackage{amsmath, amsfonts, amssymb}
\usepackage[a4paper, margin=1in]{geometry}
\usepackage{varwidth}

\title{Random Variables and Stochastic Process (AI5030)}
\author{Soumyajit Chatterjee\\AI22MTECH02005 }


\begin{document}

\maketitle

\section*{Question 56 (2019)}
There are two sets of observations on a random vector $(X,Y)$. Consider a simple linear regression model with an intercept for regressing $Y$ on $X$. Let $\hat{\beta_i}$ be the least square estimate of the regression coefficient obtained from the ith (i=1, 2) set consisting of $n_i$ observations $(n_1, n_2) > 2$. Let $\hat{\beta_0}$ be the least square  estimate obtained from the pooled sample size $n_1+n_2$. If it is known that $\hat{\beta_1} > \hat{\beta_2} > 0$, which of the following statements is true ?
\vspace{1cm}

1. $\hat{\beta_2} < \hat{\beta_0} < \hat{\beta_1}$

2. $\hat{\beta_0}$ \text{ may lie outside } $(\hat{\beta_2}, \hat{\beta_1})$ but cannot exceed $\hat{\beta_1} + \hat{\beta_2}$

3. $\hat{\beta_0}$ \text{ may lie outside } $(\hat{\beta_2}, \hat{\beta_1})$ but cannot be negative

4. $\hat{\beta_0}$ can be negative
\vspace{1cm}

\section*{Solution}
Let the straight line estimated on a vector $(X,Y)$ of sample size $n_1$, $n_2$ and $n_1+n_2$ be:

\begin{equation}
y_1 = \hat{\beta_1}\, x_1
\end{equation}
\begin{equation}
y_2 = \hat{\beta_2}\, x_2
\end{equation}
\begin{equation}
y = \hat{\beta_0}\, x
\end{equation}

\noindent The linear regression coefficient on the vector $(X,Y)$ is given by the formula:

$$
\hat{\beta} = (X^T\,X)^{-1}\,(X^T\,Y)
$$

\noindent The term $(X^TX)^{-1}$ can be written as the square of $L_2$ norm of X. Therefore, the above equation can be re-written as:

$$
\hat{\beta} = \dfrac{X^T\,Y}{||X||^2}
$$
\clearpage
\noindent Therefore, from equation 1 and 2 we can write that:

\begin{equation}
\hat{\beta_1} = \dfrac{X_1^T Y_1}{||X_1||^2} = \dfrac{p_1}{q_1}
\end{equation}

\begin{equation}
\hat{\beta_2} = \dfrac{X_2^T Y_2}{||X_2||^2} = \dfrac{p_2}{q_2}
\end{equation}

\noindent Since, $\hat{\beta_0}$ is the estimator for stacked sample size $n_1+n_2$ and can be given by:

\begin{equation}
\hat{\beta_0} = \dfrac{[X_1^T\, X_2^T] \begin{bmatrix} Y_1\\Y_2 \end{bmatrix}}{||X_1||^2 + ||X_2||^2}
\end{equation}

\noindent Upon simplification we get:

\begin{equation}
\hat{\beta_0} = \dfrac{{}X_1^T\,Y_1\, + \, X_1^T\,Y_2 \, + X_2^T\,Y_1 \, + \, X_2^T\, Y_2}{||X_1||^2 + ||X_2||^2}
\end{equation}

\vspace{3mm}
\noindent Since, all the terms in the numerator are scalars, we can represent the above equation as:

$$
\hat{\beta_0} = \dfrac{p_1 + p_2 + r + s}{q_1 + q_2}
$$

\noindent Where r is given as $X_2^T\,Y_1$ and s id given as $X_1^T\,Y_2$

\noindent In the question it is given that $\hat{\beta_1} > \hat{\beta_2} > 0$ which implies:
\vspace{4mm}

$$
\dfrac{p_1}{q_1} > \dfrac{p_2}{q_2} > 0
$$
\vspace{2mm}

\noindent Since, $q_1$ and $q_2$ is the square of norm, it is always positive. Therefore, from the above equation we can say that $p_1>0$ and $p_2>0$.

\noindent In the expression for $\hat{\beta_0}$,

$$
\hat{\beta_0} = \dfrac{p_1 + p_2 + r + s}{q_1 + q_2}
$$

\noindent it can be greater than $\hat{\beta_1}$ and $\hat{\beta_2}$ if r and s are also positive. Therefore, option (1) is incorrect.

\noindent If we evaluate the expression of $\hat{\beta_1} + \hat{\beta_2}$ it comes out to be:

$$
\hat{\beta_1} + \hat{\beta_2} = \dfrac{p_1 + p_2}{q_1 + q_2}
$$

\noindent But the equation for $\hat{\beta_0}$ is:

$$
\hat{\beta_0} = \dfrac{p_1 + p_2 + r + s}{q_1 + q_2}
$$

\noindent Therefore, 

$$
\dfrac{p_1 + p_2 + r + s}{q_1 + q_2} > \dfrac{p_1 + p_2}{q_1 + q_2}
$$

\noindent only if r and s take positive values. Therefore, option (2) is incorrect.

\noindent In the expression of $\hat{\beta_0}$ 
$$
\hat{\beta_0} = \dfrac{p_1 + p_2 + r + s}{q_1 + q_2}
$$

\noindent if r and s are negative then the numerator of the above expression can be negative. Therefore, option (3) is also incorrect. 

\noindent So, from the above conclusions drawn we can say that option (4) is the correct option.



\end{document}
